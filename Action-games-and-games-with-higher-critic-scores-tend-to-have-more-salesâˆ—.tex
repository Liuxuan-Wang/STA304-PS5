% Options for packages loaded elsewhere
\PassOptionsToPackage{unicode}{hyperref}
\PassOptionsToPackage{hyphens}{url}
%
\documentclass[
]{article}
\usepackage{lmodern}
\usepackage{amssymb,amsmath}
\usepackage{ifxetex,ifluatex}
\ifnum 0\ifxetex 1\fi\ifluatex 1\fi=0 % if pdftex
  \usepackage[T1]{fontenc}
  \usepackage[utf8]{inputenc}
  \usepackage{textcomp} % provide euro and other symbols
\else % if luatex or xetex
  \usepackage{unicode-math}
  \defaultfontfeatures{Scale=MatchLowercase}
  \defaultfontfeatures[\rmfamily]{Ligatures=TeX,Scale=1}
\fi
% Use upquote if available, for straight quotes in verbatim environments
\IfFileExists{upquote.sty}{\usepackage{upquote}}{}
\IfFileExists{microtype.sty}{% use microtype if available
  \usepackage[]{microtype}
  \UseMicrotypeSet[protrusion]{basicmath} % disable protrusion for tt fonts
}{}
\makeatletter
\@ifundefined{KOMAClassName}{% if non-KOMA class
  \IfFileExists{parskip.sty}{%
    \usepackage{parskip}
  }{% else
    \setlength{\parindent}{0pt}
    \setlength{\parskip}{6pt plus 2pt minus 1pt}}
}{% if KOMA class
  \KOMAoptions{parskip=half}}
\makeatother
\usepackage{xcolor}
\IfFileExists{xurl.sty}{\usepackage{xurl}}{} % add URL line breaks if available
\IfFileExists{bookmark.sty}{\usepackage{bookmark}}{\usepackage{hyperref}}
\hypersetup{
  pdftitle={Action games and games with higher critic scores tend to have more sales},
  pdfauthor={Liuxuan Wang},
  hidelinks,
  pdfcreator={LaTeX via pandoc}}
\urlstyle{same} % disable monospaced font for URLs
\usepackage[margin=1in]{geometry}
\usepackage{graphicx,grffile}
\makeatletter
\def\maxwidth{\ifdim\Gin@nat@width>\linewidth\linewidth\else\Gin@nat@width\fi}
\def\maxheight{\ifdim\Gin@nat@height>\textheight\textheight\else\Gin@nat@height\fi}
\makeatother
% Scale images if necessary, so that they will not overflow the page
% margins by default, and it is still possible to overwrite the defaults
% using explicit options in \includegraphics[width, height, ...]{}
\setkeys{Gin}{width=\maxwidth,height=\maxheight,keepaspectratio}
% Set default figure placement to htbp
\makeatletter
\def\fps@figure{htbp}
\makeatother
\setlength{\emergencystretch}{3em} % prevent overfull lines
\providecommand{\tightlist}{%
  \setlength{\itemsep}{0pt}\setlength{\parskip}{0pt}}
\setcounter{secnumdepth}{-\maxdimen} % remove section numbering

\title{Action games and games with higher critic scores tend to have more sales\thanks{\url{https://github.com/Liuxuan-Wang/STA304-PS5}}}
\usepackage{etoolbox}
\makeatletter
\providecommand{\subtitle}[1]{% add subtitle to \maketitle
  \apptocmd{\@title}{\par {\large #1 \par}}{}{}
}
\makeatother
\subtitle{\url{https://github.com/Liuxuan-Wang/STA304-PS5}}
\author{Liuxuan Wang}
\date{23 December 2020}

\begin{document}
\maketitle
\begin{abstract}
Video game, as an entertainment full of excitement and positive emotion,
has attracted nearly 40\% of the world population to be video game
players. Obtained video game sales data from Kaggle with over 16,000
video games, this paper conducts multiple linear regression models to
see how factors such as number of years since publish, critics and
players' evaluation and type of game affect the global video game sales,
and get the result that action games and games with higher critic scores
tend to have more global sales. From this study, video game producers
can have a better sense on what kind of games tend to have a better sale
and can predict a certain video game's sale in the future.\\
~\\
\textbf{Keywords:} video games, global sales, multiple linear regression
model
\end{abstract}

\hypertarget{introduction}{%
\section{Introduction}\label{introduction}}

Playing game is always an attractive activity for young and old. Because
of the enjoyment and positive emotion brought by the games, the love for
the games includes both inner psychological motivation and the outer
positive rewards Pang (2019). With the development of technology, video
games can provide players with more vivid visual effects, better
auditory effects and various of game machanics. As the types of games
booming, the population of players is also surging. The recent survey
has revealed that nearly 3.1 billion people around the world are video
game players, which takes up about 40\% of the world population Price
and writer (2020). In this large market, producers are pursuing to write
popular games and players with various of demand are seeking more
interesting games. For the producer side, it is important to have an
idea on which types of games are more welcomed by the consumers and the
expect sales of certain games that you have designed. Therefore, in this
paper, I am going to use R as programming tool R Core Team (2020) and
statistical method to analyze the factors that might affect the video
games sales and build up models that can predict certain types of games'
sale. The packages used are tidyverse Wickham et al. (2019), janitor
Firke (2020) and pROC Robin et al. (2011)

The data set used in this paper is from Kaggle Kirubi (2016). It
includes data about over 16,000 games published from 1980 to 2020. For
each observation, the information of platform, year of publish, genre,
publisher, sales in different area in the world, worldwide sales, and
user scores are shown in the data. In this data, I will mainly focus on
the worldwide sales to see how these factors will affect the total sales
of the video game around the world. Overall, from the multiple linear
regression we build, action games and games with higher critic scores
tend to have more global sales.

The paper will be followed a full analysis of data. In the Data section,
the data will be carefully described and evaluated. The model built up
based on the data will be shown in the Model section. Some discussions
results of the model, as well as the limitation and next steps will be
shown in the following.

\hypertarget{data}{%
\section{Data}\label{data}}

This original data was obtained from vrchartz, with 16719 games released
from 1980 to 2020. It includes the data about the games name, platform,
year of release, publisher, sales in different areas, rating and user's
and critic's score. The sales are in millions of unit. Critic's score
are in 100 while users' score are in 10. The critic's score are from
Metactrc's staff and user's scores are from Metacritic's subscriber

\hypertarget{data-features-and-strengths}{%
\subsubsection{Data Features and
Strengths}\label{data-features-and-strengths}}

\begin{itemize}
\tightlist
\item
  Data Completeless:
\end{itemize}

With 16719 games of 12 different genres, the data set has collected
almost all the games from 1980 to 2016. The variables included also
cover most of the features that can be used to evaluate a game.

\begin{itemize}
\tightlist
\item
  Broad Time span:
\end{itemize}

It has recorded the video games across over thirty years, which can give
data users' a view of video games' history using data.

\hypertarget{data-weakness}{%
\subsubsection{Data Weakness}\label{data-weakness}}

\begin{itemize}
\tightlist
\item
  Too many NA values:
\end{itemize}

Since there are a lot of old games and some of details of games are
unavailable, therefore there are many NA values in this data set.

\begin{itemize}
\tightlist
\item
  The scores are from only one platform.
\end{itemize}

Though Metacritic is one of the largest platform for evaluating games
and films, however, the scores collected here are only face to
subscribers and staffs on this platform while cannot reflect what the
other people think. There can be sampling error happening here.

\hypertarget{data-description}{%
\subsubsection{Data Description}\label{data-description}}

\includegraphics{Action-games-and-games-with-higher-critic-scores-tend-to-have-more-sales∗_files/figure-latex/unnamed-chunk-2-1.pdf}

From 1980 to 2020, in this data set, action genre has the largest number
of games released, with over 3500. Puzzle genre has only around 500
games released in this time period, which is the least.

\begin{verbatim}
##                         Name Sales Publisher Year Player_score Critic_score
## 1                 Wii Sports 82.53  Nintendo 2006            8           76
## 2          Super Mario Bros. 40.24  Nintendo 1985                        NA
## 3             Mario Kart Wii 35.52  Nintendo 2008          8.3           82
## 4          Wii Sports Resort 32.77  Nintendo 2009            8           80
## 5   Pokemon Red/Pokemon Blue 31.37  Nintendo 1996                        NA
## 6                     Tetris 30.26  Nintendo 1989                        NA
## 7      New Super Mario Bros. 29.80  Nintendo 2006          8.5           89
## 8                   Wii Play 28.92  Nintendo 2006          6.6           58
## 9  New Super Mario Bros. Wii 28.32  Nintendo 2009          8.4           87
## 10                 Duck Hunt 28.31  Nintendo 1984                        NA
\end{verbatim}

The top 10 games with the most sales are mostly released before 2010.
All of them are from Nintendo and released before 2010. From here, we
can see that these games have been popular for many years. Except for
Wii Play and some old games without critic score records, the other
games have good evaluation from both players and critics.

\includegraphics{Action-games-and-games-with-higher-critic-scores-tend-to-have-more-sales∗_files/figure-latex/unnamed-chunk-4-1.pdf}

The scatter plot above has no obvious linear pattern between these two
variables. It shows that the global sales of video games are not simply
affected by the users' evaluation. There are also a lot of other
factors.

\includegraphics{Action-games-and-games-with-higher-critic-scores-tend-to-have-more-sales∗_files/figure-latex/unnamed-chunk-5-1.pdf}

\begin{verbatim}
##    Year Frequency
## 29 2008      1427
## 30 2009      1426
## 31 2010      1255
## 28 2007      1197
## 32 2011      1136
## 27 2006      1006
## 26 2005       939
## 23 2002       829
## 24 2003       775
## 25 2004       762
## 33 2012       653
## 36 2015       606
## 35 2014       581
## 34 2013       544
## 37 2016       502
## 22 2001       482
## 19 1998       379
## 21 2000       350
## 20 1999       338
## 18 1997       289
## 40  N/A       269
## 17 1996       263
## 16 1995       219
## 15 1994       121
## 14 1993        62
## 2  1981        46
## 13 1992        43
## 12 1991        41
## 3  1982        36
## 7  1986        21
## 4  1983        17
## 10 1989        17
## 8  1987        16
## 11 1990        16
## 9  1988        15
## 5  1984        14
## 6  1985        14
## 1  1980         9
## 38 2017         3
## 39 2020         1
\end{verbatim}

From the graph and frequency table above, in this data set, 2002-2011
have the most number of games released being recorded. There are 269
games which have unavailable year of release. In 1980s, the video game
industry has just started, which leads to less games released. For games
after 2016, this data set has its limitation on recording.

\hypertarget{model}{%
\section{Model}\label{model}}

I firstly clean the raw data. In this paper, I will mainly focus on
building model predicting the global sales using year of release, genre,
critic score and user score as explanatory variables. After selecting
these variables and excluding NA values, using simple random selection
to randomly select 1000 samples from the cleaned data. And build up a
multiple regression model to see how these factors affect the global
sales. The model's summary is shown below.

\begin{verbatim}
## Warning: NAs introduced by coercion
\end{verbatim}

\begin{verbatim}
## 
## Call:
## lm(formula = Global_Sales ~ ., data = SRSdata)
## 
## Residuals:
##     Min      1Q  Median      3Q     Max 
## -4.9264 -0.5924 -0.2241  0.2139 11.1542 
## 
## Coefficients:
##                      Estimate Std. Error t value Pr(>|t|)    
## (Intercept)          2.462953   0.920162   2.677 0.007562 ** 
## Year_of_Release1997  2.572770   1.127489   2.282 0.022714 *  
## Year_of_Release1998 -3.699695   1.004719  -3.682 0.000244 ***
## Year_of_Release1999 -3.804261   1.068935  -3.559 0.000390 ***
## Year_of_Release2000 -3.941566   0.934991  -4.216 2.72e-05 ***
## Year_of_Release2001 -3.351865   0.898082  -3.732 0.000201 ***
## Year_of_Release2002 -3.855151   0.886375  -4.349 1.51e-05 ***
## Year_of_Release2003 -3.749320   0.886618  -4.229 2.57e-05 ***
## Year_of_Release2004 -3.656148   0.887438  -4.120 4.12e-05 ***
## Year_of_Release2005 -3.910140   0.887176  -4.407 1.16e-05 ***
## Year_of_Release2006 -3.894499   0.886035  -4.395 1.23e-05 ***
## Year_of_Release2007 -3.739342   0.884490  -4.228 2.59e-05 ***
## Year_of_Release2008 -3.593241   0.882693  -4.071 5.07e-05 ***
## Year_of_Release2009 -3.819187   0.885167  -4.315 1.76e-05 ***
## Year_of_Release2010 -3.557041   0.887200  -4.009 6.56e-05 ***
## Year_of_Release2011 -3.429815   0.889478  -3.856 0.000123 ***
## Year_of_Release2012 -4.003999   0.894907  -4.474 8.58e-06 ***
## Year_of_Release2013 -4.039699   0.897865  -4.499 7.65e-06 ***
## Year_of_Release2014 -3.910451   0.894237  -4.373 1.36e-05 ***
## Year_of_Release2015 -3.540132   0.891354  -3.972 7.67e-05 ***
## Year_of_Release2016 -4.355701   0.903789  -4.819 1.67e-06 ***
## GenreAdventure      -0.330803   0.233033  -1.420 0.156060    
## GenreFighting       -0.026169   0.190593  -0.137 0.890820    
## GenreMisc            0.169198   0.181481   0.932 0.351406    
## GenrePlatform        0.092306   0.186892   0.494 0.621488    
## GenrePuzzle         -0.466466   0.311525  -1.497 0.134626    
## GenreRacing          0.083241   0.154900   0.537 0.591126    
## GenreRole-Playing   -0.167019   0.145772  -1.146 0.252179    
## GenreShooter         0.061925   0.136397   0.454 0.649929    
## GenreSimulation     -0.158294   0.205477  -0.770 0.441265    
## GenreSports         -0.239212   0.134708  -1.776 0.076083 .  
## GenreStrategy       -0.641281   0.228396  -2.808 0.005089 ** 
## Critic_Score         0.041340   0.003726  11.094  < 2e-16 ***
## Player_Score        -0.112621   0.037682  -2.989 0.002872 ** 
## ---
## Signif. codes:  0 '***' 0.001 '**' 0.01 '*' 0.05 '.' 0.1 ' ' 1
## 
## Residual standard error: 1.227 on 966 degrees of freedom
## Multiple R-squared:  0.2465, Adjusted R-squared:  0.2208 
## F-statistic: 9.577 on 33 and 966 DF,  p-value: < 2.2e-16
\end{verbatim}

\hypertarget{results}{%
\section{Results}\label{results}}

From the summary of multiple linear regression above, most of genre
variables' coefficient are not statistically significant. For year of
release, the year of 1996 is the base year, and for genre, action is the
base variable. The coefficients for year of release show that the global
sales will be more when the time of release becomes longer. The
coefficients for genre just align with the situation we observe from the
data part, that action games tend to have more sales. Critic score and
player's evaluation both have siginificant affect on the global sales.
Critic score has statistically significant positive relationship with
globals sales. Surprisingly, the relationship between player's score and
global sales is negative. Commonly, we think that the game with higher
player evaluation should have more sales. This can be the effect of
sampling.

\hypertarget{discussion}{%
\section{Discussion}\label{discussion}}

\begin{enumerate}
\def\labelenumi{\arabic{enumi})}
\item
  For some video games such as Mario and Pokemon, though they have been
  released for over 20 or 30 years, they are still populate and
  favorited by players from all over the world. Part of the reasons that
  why the coefficients for year of release are extremely close can be
  generated from these extremely popular old games which have larger
  sales growth than the other later games.
\item
  The circumstance shown in the coefficients of player score and critic
  score can be reasonable. Since players can be attracted by the high
  scores and pay for the game. However, player's evaluations mostly come
  out after they buy it. Therefore, the relationship between player's
  evaluation and the games' sales can be negative.
\end{enumerate}

\hypertarget{limitation}{%
\subsection{Limitation}\label{limitation}}

\begin{itemize}
\tightlist
\item
  Sampling method
\end{itemize}

The sampling method used here is simple random selection, which can have
some biases on sampling.

\begin{itemize}
\tightlist
\item
  Other factors
\end{itemize}

Except for the factors we used in this report, there are still other
factors that can affect the sales. For example, for games on platforms
that can expose to more players, the sales tend to be higher. Nintendo
as large factory of video games starting in early years, tend to own
more high sales games, as well.

\hypertarget{next-steps}{%
\subsection{Next steps}\label{next-steps}}

\begin{itemize}
\tightlist
\item
  Use more advanced sampling method:
\end{itemize}

Using stratified sampling method according to the platform or the genre
of games can be a better choice with less biases than simple linear
regression.

\begin{itemize}
\tightlist
\item
  Find more complete data from different platforms
\end{itemize}

Using data from different evaluation platforms can generate less bias on
sampling error.

\hypertarget{reference}{%
\section*{Reference}\label{reference}}
\addcontentsline{toc}{section}{Reference}

\hypertarget{refs}{}
\leavevmode\hypertarget{ref-janitor}{}%
Firke, Sam. 2020. \emph{Janitor: Simple Tools for Examining and Cleaning
Dirty Data}. \url{https://CRAN.R-project.org/package=janitor}.

\leavevmode\hypertarget{ref-kirubi_2016}{}%
Kirubi, Rush. 2016. ``Video Game Sales with Ratings.''
\url{https://www.kaggle.com/rush4ratio/video-game-sales-with-ratings}.

\leavevmode\hypertarget{ref-pang_2019}{}%
Pang, Chirlien. 2019. ``Understanding Gamer Psychology: Why Do People
Play Games?'' \emph{Sekg}.
\url{https://www.sekg.net/gamer-psychology-people-play-games/}.

\leavevmode\hypertarget{ref-price_ben}{}%
Price, Ben, and Ben Price(130 Articles Published)Ben is a writer. 2020.
``New Report Shows Percentage of Global Population That Plays Video
Games.'' \emph{Game Rant}.
\url{https://gamerant.com/3-billion-gamers-report/}.

\leavevmode\hypertarget{ref-citeR}{}%
R Core Team. 2020. \emph{R: A Language and Environment for Statistical
Computing}. Vienna, Austria: R Foundation for Statistical Computing.
\url{https://www.R-project.org/}.

\leavevmode\hypertarget{ref-pROC}{}%
Robin, Xavier, Natacha Turck, Alexandre Hainard, Natalia Tiberti,
Frédérique Lisacek, Jean-Charles Sanchez, and Markus Müller. 2011.
``PROC: An Open-Source Package for R and S+ to Analyze and Compare Roc
Curves.'' \emph{BMC Bioinformatics} 12: 77.

\leavevmode\hypertarget{ref-tidyverse}{}%
Wickham, Hadley, Mara Averick, Jennifer Bryan, Winston Chang, Lucy
D'Agostino McGowan, Romain François, Garrett Grolemund, et al. 2019.
``Welcome to the tidyverse.'' \emph{Journal of Open Source Software} 4
(43): 1686. \url{https://doi.org/10.21105/joss.01686}.

\end{document}
